\documentclass{beamer}

%Include Packages
\usepackage{amsmath}
\usepackage{amssymb}
\usepackage[normalem]{ulem}		%For underline
\usepackage{graphicx}
\usepackage{tikz}

%\usepackage{beamerthemesplit}
%\usetheme{Berkeley}
%\usecolortheme{dolphin}
\logo{\includegraphics[height=0.6cm]{logo}}

%Top Matter
\title{A Monte Carlo Study of the Classical, Isotropic, 3D Heisenberg Model}
\subtitle{Numerical Studies of Stochastic Spin Systems}
\author{Michael Conroy\\
  PHY 471 Capstone Project \\
  Spring 2014 \\
  Professor: Dr. Matthew Enjalran}
\date{May 2, 2014}

\begin{document}
	%Title Frame
	\begin{frame}
	%\frametitle{}
	\titlepage
	\end{frame}

	%Table of Contents Frame
	\begin{frame}
		\frametitle{Table of Contents}
		\tableofcontents
	\end{frame}

	\section{Introduction}
	
	\subsection{Goal and Purpose}
  \begin{frame}
    \frametitle{Goal and Purpose}
    	\begin{itemize}
    		\item Simulate the classical, isotropic, 3D Heisenberg Model on the simple cubic lattice
    		\item Utilize the Monte Carlo method with the Metropolis Algorithm
    		\item Compare simulation data to literature data
    	\end{itemize}
  \end{frame}
  
  \subsection{Brief Overview}
  \begin{frame}
    \frametitle{Brief Overview}
    	\begin{itemize}
    		We undertook a study of stochastic spin system during the 2013-2014 school year for a PHY 499 Independent Study and PHY 471 Capstone Project. The fields of thermodynamics, statistical mechanics, and numerical methods are applied to compare analytic solutions to Monte Carlo Metropolis simulations of the 3-State Problem, the Ising Paramagnet, and the Heisenberg Paramagnet. Systems without analytic solutions are numerically explored and compared with current research data. The 2D and 3D Ising Model are explored in preparation for simulations of the classical Heisenberg Model. Finally, the classical Heisenberg Model is discussed, simulations presented, and results reviewed. 

    	\end{itemize}
  \end{frame}

	\subsection{Background and Theory}
	\subsubsection{Statistical Mechanics}
  \begin{frame}
    \frametitle{Background and Theory}
    \framesubtitle{}
    \begin{itemize}
    	\item Equilibrium Statistical Mechanics
    	\item Canonical Ensemble
    	\item Boltzmann Distribution
    	\item Partition Function
    	\item Energy, specific heat, entropy, free energy
    	\item Fluctuations
    \end{itemize}
  \end{frame}
  \subsubsection{The Heisenberg Model}
  \begin{frame}
    \frametitle{Background and Theory}
    \framesubtitle{}
  \end{frame}
  \subsubsection{Monte Carlo Method}
  \begin{frame}
    \frametitle{Background and Theory}
    \framesubtitle{}
    \begin{itemize}
    	\item Numerical Analysis
    	\begin{itemize}
    		\item No analytic solution or intractable
    	\end{itemize}
    	\item Monte Carlo Simulation
    	\begin{itemize}
    		\item Estimator
    		\item Importance Sampling
    		\begin{itemize}
    			\item Markov Processes
    			\item Ergodicity
    			\item Detailed Balance
    		\end{itemize}
    		\item Acceptance Ratio
    	\end{itemize}
    \end{itemize}
  \end{frame}
  \begin{frame}
    \frametitle{Background and Theory}
    \framesubtitle{Monte Carlo Example}
    \item Pi calculation example (?)
  \end{frame}

	\section{Method}  
  %Frame 5
  \begin{frame}
    \frametitle{Directly Related Stuff}
    %Content goes here
  \end{frame}
  
  %Frame 6
  \begin{frame}
    \frametitle{Steps in Chronological Order}
    \framesubtitle{A bit more information about this}
    %More content goes here
  \end{frame}
    
  \section{Results}
  %Frame 7
  \begin{frame}
    \frametitle{Results}
    %Content goes here
  \end{frame}
  
  \section{Discussion and Conclusion}
  %Frame 8
  \begin{frame}
    \frametitle{Conclusions to be Drawn}
    \framesubtitle{A bit more information about this}
    %More content goes here
  \end{frame}
  
  %Frame 9
  \begin{frame}
    \frametitle{Ways to Improve/Issues}
    %Content goes here
  \end{frame}
  
  \section{Future Work}
  %Frame 10
  \begin{frame}
    \frametitle{Next Steps}
    \framesubtitle{A bit more information about this}
    %More content goes here
  \end{frame}
  
  %Frame 11
  \begin{frame}
    \frametitle{This is the first slide}
    %Content goes here
  \end{frame}
  
  %Frame 12
  \begin{frame}
    \frametitle{This is the second slide}
    \framesubtitle{A bit more information about this}
    %More content goes here
  \end{frame}

\end{document}